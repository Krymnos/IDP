\section{Provenance API endpoints}

In this backend section, we will be explaining about the Provenance API and the technologies used to develop the same. Provenance API is developed mainly as an intermediary between the provenance Database and the frontend.
The frontend will be issuing the GET requests to the API endpoints defined by our Provenance API backend.

\subsection{Technologies Used}

    \begin{itemize}
        \item Developed in Java
        \item Maven project
        \item Spring-Boot
        \item Configuration location using environment variables
        \item Cassandra integration
    \end{itemize}

\subsection{API Endpoints}

We will be explaining the endpoints and the information provided by them,

\subsubsection{/cluster/stats}

To present the visualization of the current running system to the end users, the front end will be querying this /cluster/stats endpoint and get the real time status about each nodes and also the health rate about the nodes. 

\subsubsection{/cluster/topology}

The end users will also be presented with the graph tree like structure of the running provenance system. This visualization will show the each node and also the edges pointing to the next successor node. This endpoint provides a list nodes and their successor in JSON format to the querying frontend. 

\subsubsection{/cluster/{nodeId}/provenance}

This endpoint provides a JSON response about all the provenance datapoints that are specific to a particular node in the provenance system. In the frontend, the end user will be clicking over the node presented in the visualized topology, and a table will be presented below with the list of provenance data pertaining to that particular node.

\subsubsection{/provenance/{id}}

This endpoint provides the all datapoints that contributed to the creation of this particular datapoint. The field inputDatapoint provides that information. JSON response contains a nested loop like structure which embeds all the datapoints related to this datapoint. 

\subsubsection{/provenance/{id}/static}

This endpoint provides the all datapoints that contributed to the creation of this particular datapoint, but in a downloadable JSON format.

\subsection{Swagger API documentation}

After we have defined all the endpoints in the backend, we have used the Swagger API to provide a easily communicable way of explanation about our endpoints. 

Moreover, every change in the API should be simultaneously updated in the Swagger API documentation. Accomplishing this manually is a tedious process, so automation of the process was inevitable. To provide this automation we have directly integrated swagger configuration to the spring boot application.

We have used the Docket plugin for the spring boot and defined the configuration by creating a SwaggerConfig.java file and once building this changes will enable a new endpoint /swagger-ui.html , this file provides required Swagger API documentation.



\section{Provenance Database}

Database is at the heart of every architecture especially when there a focus on the data. Choosing the right database goes a long way and the effect of this will be shown in each and every component of the architecture. In IoT domain where the data ingestion is huge and the data needs to query in real-time having a particular set of queries from where the user can query the dataset makes the decision more difficult.

In order to come up with the suitable database we underlying few basic factors which play an important role in finding the perfect match for the our system.

\paragraph*{Size of data to be stored:}
This factor considers amount of data stored and retrieved. So we choose the database keeping in mind the overall volume of data generated by the application at any given time and also the size of the data which is retrieved can handle efficiently.

\paragraph*{Speed and Scalability:}
This factor considers the speed of reading data from the database and writing data to the database. Some database focuses more on the read-heavy application, while others are designed to cope with the write-heavy operation on the application. This factor is important to our case as more data is writen to the database when adding new resources.

\paragraph*{Accessibility of data:}
Accessibility of data is important as the user of the system need to access the information as soon as possible it is stored for querying. The database needs to handle concurrent writes by numerous component in the architecture connected to the database. The effect of a huge load of writes should be minimum and there should be a mechanism in place which handles fault tolerance in the database. This will ensure all the provenance information is safe and is not lost forever.

\paragraph*{Data modeling:}
The structure is the core component in choosing the right database for the application. As the data from the sensors and the provenance information vary a lot. The data model is not specifically fixed in hard boundaries but can be capped under major categories. The database must be able to handle changes in the data model and adapt quickly.

\paragraph*{Safety and security of data:}
Storing information about the data from the sensor deployed on the ground capture various metrics which are confidential and contains information which is critical. In order to preserve the confidentiality and secrecy, the database must handle some level of security. The safety measures implemented by the database in case of any system crash or failure is quite a significant factor to keep in mind while choosing a database especially in our case where data lost means permanent loss of information.

\subsection{Apache Cassandra as Provenance Database}

After thoroughly considering a numerous number of databases from the pool we decided to use Apache Cassandra as the Provenance Database for our system.
Apache Cassandra is an open source, distributed, massively scalable NoSQL database. It is designed to handle large volumes of structured, semi-structured and unstructured data across multiple data centers, and it supports the cloud deployment. Cassandra offers capabilities like continuous availability, linear scalability and operational simplicity across many commodity servers with no single point of failure. Its powerful, dynamic data model is designed for maximum flexibility and fast response times. 

Apache Cassandra supports configured consistency levels to manage availability versus data accuracy for writes-heavy demand in IoT domain.  Data is compressed up to 80 percent without any performance overhead this can lead to storing more volume of data in less amount of space. 
Data is distributed across the cluster and there is no master node in the system so each node can service any request. The distributed architecture is perfect for disaster recovery, redundancy and failover as the data are different to create from the start.It can handle massive data sizes and scale out to large clusters. 

Apache Cassandra offering continuous availability, high scalability and performance, strong security, and operational simplicity. It has flexible data storage which easily accommodates the data in various formats and structure. Changes can be made dynamically to the data structure as per requirement.


Apache Cassandra outperformed other NoSQL database in the benchmark using YSCB and was also run on the sample dataset of our data model and performed fairly in term of writes. 
Apache Cassandra is the best choice for our scenario as it fulfills all the of providing a high write performance with a high load of receiving a message from a different component of the pipeline and the data is quickly available for the user to query. For IoT domain scenario, most of the nodes are distributed in numerous location so deployment of Apache Cassandra at different geographically distributed region could help in increasing the performance and also for faster retrieval of data.

\paragraph*{Apache Cassandra Query Language}
The Cassandra Query Language (CQL) allows you to query Cassandra using queries similar to SQL.CQL commands include data definition queries (e.g., create table), data manipulation queries (e.g., insert and select for rows), and basic authentication queries to create database users and grant them permissions.This fulfills the requirement of the user querying the provenance database for retrieval of provenance information.







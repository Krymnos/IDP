\section{Data Provenance Model}
In this section, we shortly explain some of the existing provenance data models and the need for a new data model, then we explain a new data model specialized in the IOT context. Our data model is inspired by the Data Provenance Model\footnote{\url{https://link.springer.com/content/pdf/10.1007\%2F978-3-319-68136-8.pdf}}.
\lstset{stringstyle=\color{black}}


\subsection{Existing Data Models}

\subsection{New Data Model}

We define a data point (DP) as a uniquely identifiable and addressable piece of data (i.e., a value) in the context of the smart grid. Examples for DPs in the context of smart grid includes sensor readings such as 3-phase electric currents, complex analytics results derived from sensor readings etc. The unique identifier is composed of three blocks.
\begin{itemize}
	\item Unique Identifier - A data point distinguishes itself specifically from other data flowing in the Data Provenance Model for smart grid  (e.g., bulk sensor readings that are of no further interest, ephemeral intermediary analytics results, etc.) in that it is addressable, i.e., it has an ID that is unique in the context of the smart grid. In order to ensure the uniqueness of the identifier, we came up with an identifier generation strategy, which generates a unique ten-byte identifier and ensures uniqueness across the system. Data point identifier comprised of:
		\begin{itemize}
			\item 3-byte node identifier - Guarantees its uniqueness across machines/nodes and processes.
			\item 4-byte value representing the seconds since the Unix epoch - Ensures uniqueness in relation to a single second.
			\item 3-byte counter - Provides uniqueness within a single second in a single process.
		\end{itemize}
		\subsection* {Properties} Our unique identifier mechanism along with its simplicity brought some other advantages and reduced the need to store machine/node identifier separately and also ease some time-based queries (e.g., sorting based on generation timestamp).
			\begin{itemize}				
				\item Example: 5a7b91370003c6badfb2
			\end{itemize}
	\item Input Data Points - A data point may be based on other data points that have contributed to its creation or modification. We refer to these related
data points as input data points(IDP). A data point's IDP is a list of unique identifiers of all those data points, which contributed to its creation or transformation. It also stores information about the contribution type.
	
	\begin{itemize}			
			\item Example: 
\begin{lstlisting}			
[{
    "average": [   // Contribution type
        "5a81c07800031ddaf123",
        "5a81c093000a1d341fab"
    ]
}]		
\end{lstlisting}
	\end{itemize}
	\item Context - Specific context for provenance may vary for different IoT applications. We propose a data model for the context in typical IoT environments comprising the concepts of Agents, Execution Context, as well as Time and Location information (cf. \ref{dataModelContext}). An Agent is an entity that creates and/or modifies data points (e.g., sensor, device,  software agent, etc.). It is recursively defined in such a way that an agent can contain other agents (e.g., a device containing several sensors). This recursion allows for defining agents in a hierarchy and may be used as fine-grained as required. For instance, an agent hierarchy may span from the concept of a particular function in a software library running over a virtualization container on a particular device to a particular IoT network. Execution Context provides information related to the provenance event at runtimes, such as events or data points that triggered the creation/modification of the data point. Time and Location information is also added to the provenance information.
\begin{figure}[h]
\centering
\includegraphics[width=\linewidth]{figures/context.png}\\
\caption{Data Model Context}
\label{dataModelContext}
\end{figure}
We chose two to three specific metrics for each of the above-mentioned context parameters based on our use-case and these metrics are listed below  (cf. \ref{idpDataModelContext}):
		\begin{itemize}
			\item Node/Machine  identifier
			\item DP creation time
			\item DP send time
			\item DP receive time
			\item Application name
			\item Class name
			\item Location
			\item Line of code 
			\item health status
				\begin{itemize}
					\item Node/Machiene
					\item Channel
					\item Provenance Daemon
					\item Pipeline Daemon
				\end{itemize}		
		\end{itemize}
We categorize these metrics into two main categories: core metrics and use-case specific metrics. Node/Machine identifier, DP creation time, DP send time, DP receive time and Location are the core metrics of our provenance system. All other metrics are use-case specific: Application name, class name and code line number are for debugging and tracing use-case while health status is for error detection use-case.

\begin{figure}[h]
\centering
\includegraphics[width=\linewidth]{figures/contextIdp.png}\\
\caption{Use-case specific Data Model Context}
\label{idpDataModelContext}
\end{figure}
		
\begin{figure}[h]
\centering
\includegraphics[width=\linewidth]{figures/dataModelforIDP.png}\\
\caption{Data Model}
\label{dataModel}
\end{figure}
\end{itemize}
TODO